
% Default to the notebook output style

    


% Inherit from the specified cell style.




    
\documentclass{article}

    
    
    \usepackage{graphicx} % Used to insert images
    \usepackage{adjustbox} % Used to constrain images to a maximum size 
    \usepackage{color} % Allow colors to be defined
    \usepackage{enumerate} % Needed for markdown enumerations to work
    \usepackage{geometry} % Used to adjust the document margins
    \usepackage{amsmath} % Equations
    \usepackage{amssymb} % Equations
    \usepackage{eurosym} % defines \euro
    \usepackage[mathletters]{ucs} % Extended unicode (utf-8) support
    \usepackage[utf8x]{inputenc} % Allow utf-8 characters in the tex document
    \usepackage{fancyvrb} % verbatim replacement that allows latex
    \usepackage{grffile} % extends the file name processing of package graphics 
                         % to support a larger range 
    % The hyperref package gives us a pdf with properly built
    % internal navigation ('pdf bookmarks' for the table of contents,
    % internal cross-reference links, web links for URLs, etc.)
    \usepackage{hyperref}
    \usepackage{longtable} % longtable support required by pandoc >1.10
    \usepackage{booktabs}  % table support for pandoc > 1.12.2
    

    
    
    \definecolor{orange}{cmyk}{0,0.4,0.8,0.2}
    \definecolor{darkorange}{rgb}{.71,0.21,0.01}
    \definecolor{darkgreen}{rgb}{.12,.54,.11}
    \definecolor{myteal}{rgb}{.26, .44, .56}
    \definecolor{gray}{gray}{0.45}
    \definecolor{lightgray}{gray}{.95}
    \definecolor{mediumgray}{gray}{.8}
    \definecolor{inputbackground}{rgb}{.95, .95, .85}
    \definecolor{outputbackground}{rgb}{.95, .95, .95}
    \definecolor{traceback}{rgb}{1, .95, .95}
    % ansi colors
    \definecolor{red}{rgb}{.6,0,0}
    \definecolor{green}{rgb}{0,.65,0}
    \definecolor{brown}{rgb}{0.6,0.6,0}
    \definecolor{blue}{rgb}{0,.145,.698}
    \definecolor{purple}{rgb}{.698,.145,.698}
    \definecolor{cyan}{rgb}{0,.698,.698}
    \definecolor{lightgray}{gray}{0.5}
    
    % bright ansi colors
    \definecolor{darkgray}{gray}{0.25}
    \definecolor{lightred}{rgb}{1.0,0.39,0.28}
    \definecolor{lightgreen}{rgb}{0.48,0.99,0.0}
    \definecolor{lightblue}{rgb}{0.53,0.81,0.92}
    \definecolor{lightpurple}{rgb}{0.87,0.63,0.87}
    \definecolor{lightcyan}{rgb}{0.5,1.0,0.83}
    
    % commands and environments needed by pandoc snippets
    % extracted from the output of `pandoc -s`
    \DefineVerbatimEnvironment{Highlighting}{Verbatim}{commandchars=\\\{\}}
    % Add ',fontsize=\small' for more characters per line
    \newenvironment{Shaded}{}{}
    \newcommand{\KeywordTok}[1]{\textcolor[rgb]{0.00,0.44,0.13}{\textbf{{#1}}}}
    \newcommand{\DataTypeTok}[1]{\textcolor[rgb]{0.56,0.13,0.00}{{#1}}}
    \newcommand{\DecValTok}[1]{\textcolor[rgb]{0.25,0.63,0.44}{{#1}}}
    \newcommand{\BaseNTok}[1]{\textcolor[rgb]{0.25,0.63,0.44}{{#1}}}
    \newcommand{\FloatTok}[1]{\textcolor[rgb]{0.25,0.63,0.44}{{#1}}}
    \newcommand{\CharTok}[1]{\textcolor[rgb]{0.25,0.44,0.63}{{#1}}}
    \newcommand{\StringTok}[1]{\textcolor[rgb]{0.25,0.44,0.63}{{#1}}}
    \newcommand{\CommentTok}[1]{\textcolor[rgb]{0.38,0.63,0.69}{\textit{{#1}}}}
    \newcommand{\OtherTok}[1]{\textcolor[rgb]{0.00,0.44,0.13}{{#1}}}
    \newcommand{\AlertTok}[1]{\textcolor[rgb]{1.00,0.00,0.00}{\textbf{{#1}}}}
    \newcommand{\FunctionTok}[1]{\textcolor[rgb]{0.02,0.16,0.49}{{#1}}}
    \newcommand{\RegionMarkerTok}[1]{{#1}}
    \newcommand{\ErrorTok}[1]{\textcolor[rgb]{1.00,0.00,0.00}{\textbf{{#1}}}}
    \newcommand{\NormalTok}[1]{{#1}}
    
    % Define a nice break command that doesn't care if a line doesn't already
    % exist.
    \def\br{\hspace*{\fill} \\* }
    % Math Jax compatability definitions
    \def\gt{>}
    \def\lt{<}
    % Document parameters
    \title{Lorenz\%20Differential\%20Equations}
    
    
    

    % Pygments definitions
    
\makeatletter
\def\PY@reset{\let\PY@it=\relax \let\PY@bf=\relax%
    \let\PY@ul=\relax \let\PY@tc=\relax%
    \let\PY@bc=\relax \let\PY@ff=\relax}
\def\PY@tok#1{\csname PY@tok@#1\endcsname}
\def\PY@toks#1+{\ifx\relax#1\empty\else%
    \PY@tok{#1}\expandafter\PY@toks\fi}
\def\PY@do#1{\PY@bc{\PY@tc{\PY@ul{%
    \PY@it{\PY@bf{\PY@ff{#1}}}}}}}
\def\PY#1#2{\PY@reset\PY@toks#1+\relax+\PY@do{#2}}

\expandafter\def\csname PY@tok@mi\endcsname{\def\PY@tc##1{\textcolor[rgb]{0.40,0.40,0.40}{##1}}}
\expandafter\def\csname PY@tok@se\endcsname{\let\PY@bf=\textbf\def\PY@tc##1{\textcolor[rgb]{0.73,0.40,0.13}{##1}}}
\expandafter\def\csname PY@tok@nf\endcsname{\def\PY@tc##1{\textcolor[rgb]{0.00,0.00,1.00}{##1}}}
\expandafter\def\csname PY@tok@mo\endcsname{\def\PY@tc##1{\textcolor[rgb]{0.40,0.40,0.40}{##1}}}
\expandafter\def\csname PY@tok@nc\endcsname{\let\PY@bf=\textbf\def\PY@tc##1{\textcolor[rgb]{0.00,0.00,1.00}{##1}}}
\expandafter\def\csname PY@tok@il\endcsname{\def\PY@tc##1{\textcolor[rgb]{0.40,0.40,0.40}{##1}}}
\expandafter\def\csname PY@tok@gp\endcsname{\let\PY@bf=\textbf\def\PY@tc##1{\textcolor[rgb]{0.00,0.00,0.50}{##1}}}
\expandafter\def\csname PY@tok@si\endcsname{\let\PY@bf=\textbf\def\PY@tc##1{\textcolor[rgb]{0.73,0.40,0.53}{##1}}}
\expandafter\def\csname PY@tok@kd\endcsname{\let\PY@bf=\textbf\def\PY@tc##1{\textcolor[rgb]{0.00,0.50,0.00}{##1}}}
\expandafter\def\csname PY@tok@gd\endcsname{\def\PY@tc##1{\textcolor[rgb]{0.63,0.00,0.00}{##1}}}
\expandafter\def\csname PY@tok@w\endcsname{\def\PY@tc##1{\textcolor[rgb]{0.73,0.73,0.73}{##1}}}
\expandafter\def\csname PY@tok@gt\endcsname{\def\PY@tc##1{\textcolor[rgb]{0.00,0.27,0.87}{##1}}}
\expandafter\def\csname PY@tok@gh\endcsname{\let\PY@bf=\textbf\def\PY@tc##1{\textcolor[rgb]{0.00,0.00,0.50}{##1}}}
\expandafter\def\csname PY@tok@ow\endcsname{\let\PY@bf=\textbf\def\PY@tc##1{\textcolor[rgb]{0.67,0.13,1.00}{##1}}}
\expandafter\def\csname PY@tok@ni\endcsname{\let\PY@bf=\textbf\def\PY@tc##1{\textcolor[rgb]{0.60,0.60,0.60}{##1}}}
\expandafter\def\csname PY@tok@s\endcsname{\def\PY@tc##1{\textcolor[rgb]{0.73,0.13,0.13}{##1}}}
\expandafter\def\csname PY@tok@gu\endcsname{\let\PY@bf=\textbf\def\PY@tc##1{\textcolor[rgb]{0.50,0.00,0.50}{##1}}}
\expandafter\def\csname PY@tok@sb\endcsname{\def\PY@tc##1{\textcolor[rgb]{0.73,0.13,0.13}{##1}}}
\expandafter\def\csname PY@tok@gi\endcsname{\def\PY@tc##1{\textcolor[rgb]{0.00,0.63,0.00}{##1}}}
\expandafter\def\csname PY@tok@na\endcsname{\def\PY@tc##1{\textcolor[rgb]{0.49,0.56,0.16}{##1}}}
\expandafter\def\csname PY@tok@kr\endcsname{\let\PY@bf=\textbf\def\PY@tc##1{\textcolor[rgb]{0.00,0.50,0.00}{##1}}}
\expandafter\def\csname PY@tok@m\endcsname{\def\PY@tc##1{\textcolor[rgb]{0.40,0.40,0.40}{##1}}}
\expandafter\def\csname PY@tok@mh\endcsname{\def\PY@tc##1{\textcolor[rgb]{0.40,0.40,0.40}{##1}}}
\expandafter\def\csname PY@tok@ne\endcsname{\let\PY@bf=\textbf\def\PY@tc##1{\textcolor[rgb]{0.82,0.25,0.23}{##1}}}
\expandafter\def\csname PY@tok@sd\endcsname{\let\PY@it=\textit\def\PY@tc##1{\textcolor[rgb]{0.73,0.13,0.13}{##1}}}
\expandafter\def\csname PY@tok@gr\endcsname{\def\PY@tc##1{\textcolor[rgb]{1.00,0.00,0.00}{##1}}}
\expandafter\def\csname PY@tok@kc\endcsname{\let\PY@bf=\textbf\def\PY@tc##1{\textcolor[rgb]{0.00,0.50,0.00}{##1}}}
\expandafter\def\csname PY@tok@ge\endcsname{\let\PY@it=\textit}
\expandafter\def\csname PY@tok@sx\endcsname{\def\PY@tc##1{\textcolor[rgb]{0.00,0.50,0.00}{##1}}}
\expandafter\def\csname PY@tok@err\endcsname{\def\PY@bc##1{\setlength{\fboxsep}{0pt}\fcolorbox[rgb]{1.00,0.00,0.00}{1,1,1}{\strut ##1}}}
\expandafter\def\csname PY@tok@s2\endcsname{\def\PY@tc##1{\textcolor[rgb]{0.73,0.13,0.13}{##1}}}
\expandafter\def\csname PY@tok@nn\endcsname{\let\PY@bf=\textbf\def\PY@tc##1{\textcolor[rgb]{0.00,0.00,1.00}{##1}}}
\expandafter\def\csname PY@tok@kn\endcsname{\let\PY@bf=\textbf\def\PY@tc##1{\textcolor[rgb]{0.00,0.50,0.00}{##1}}}
\expandafter\def\csname PY@tok@nd\endcsname{\def\PY@tc##1{\textcolor[rgb]{0.67,0.13,1.00}{##1}}}
\expandafter\def\csname PY@tok@vc\endcsname{\def\PY@tc##1{\textcolor[rgb]{0.10,0.09,0.49}{##1}}}
\expandafter\def\csname PY@tok@nt\endcsname{\let\PY@bf=\textbf\def\PY@tc##1{\textcolor[rgb]{0.00,0.50,0.00}{##1}}}
\expandafter\def\csname PY@tok@k\endcsname{\let\PY@bf=\textbf\def\PY@tc##1{\textcolor[rgb]{0.00,0.50,0.00}{##1}}}
\expandafter\def\csname PY@tok@sh\endcsname{\def\PY@tc##1{\textcolor[rgb]{0.73,0.13,0.13}{##1}}}
\expandafter\def\csname PY@tok@go\endcsname{\def\PY@tc##1{\textcolor[rgb]{0.53,0.53,0.53}{##1}}}
\expandafter\def\csname PY@tok@kt\endcsname{\def\PY@tc##1{\textcolor[rgb]{0.69,0.00,0.25}{##1}}}
\expandafter\def\csname PY@tok@vi\endcsname{\def\PY@tc##1{\textcolor[rgb]{0.10,0.09,0.49}{##1}}}
\expandafter\def\csname PY@tok@nb\endcsname{\def\PY@tc##1{\textcolor[rgb]{0.00,0.50,0.00}{##1}}}
\expandafter\def\csname PY@tok@s1\endcsname{\def\PY@tc##1{\textcolor[rgb]{0.73,0.13,0.13}{##1}}}
\expandafter\def\csname PY@tok@gs\endcsname{\let\PY@bf=\textbf}
\expandafter\def\csname PY@tok@kp\endcsname{\def\PY@tc##1{\textcolor[rgb]{0.00,0.50,0.00}{##1}}}
\expandafter\def\csname PY@tok@c\endcsname{\let\PY@it=\textit\def\PY@tc##1{\textcolor[rgb]{0.25,0.50,0.50}{##1}}}
\expandafter\def\csname PY@tok@nl\endcsname{\def\PY@tc##1{\textcolor[rgb]{0.63,0.63,0.00}{##1}}}
\expandafter\def\csname PY@tok@ss\endcsname{\def\PY@tc##1{\textcolor[rgb]{0.10,0.09,0.49}{##1}}}
\expandafter\def\csname PY@tok@mb\endcsname{\def\PY@tc##1{\textcolor[rgb]{0.40,0.40,0.40}{##1}}}
\expandafter\def\csname PY@tok@sr\endcsname{\def\PY@tc##1{\textcolor[rgb]{0.73,0.40,0.53}{##1}}}
\expandafter\def\csname PY@tok@cs\endcsname{\let\PY@it=\textit\def\PY@tc##1{\textcolor[rgb]{0.25,0.50,0.50}{##1}}}
\expandafter\def\csname PY@tok@sc\endcsname{\def\PY@tc##1{\textcolor[rgb]{0.73,0.13,0.13}{##1}}}
\expandafter\def\csname PY@tok@bp\endcsname{\def\PY@tc##1{\textcolor[rgb]{0.00,0.50,0.00}{##1}}}
\expandafter\def\csname PY@tok@mf\endcsname{\def\PY@tc##1{\textcolor[rgb]{0.40,0.40,0.40}{##1}}}
\expandafter\def\csname PY@tok@c1\endcsname{\let\PY@it=\textit\def\PY@tc##1{\textcolor[rgb]{0.25,0.50,0.50}{##1}}}
\expandafter\def\csname PY@tok@nv\endcsname{\def\PY@tc##1{\textcolor[rgb]{0.10,0.09,0.49}{##1}}}
\expandafter\def\csname PY@tok@no\endcsname{\def\PY@tc##1{\textcolor[rgb]{0.53,0.00,0.00}{##1}}}
\expandafter\def\csname PY@tok@o\endcsname{\def\PY@tc##1{\textcolor[rgb]{0.40,0.40,0.40}{##1}}}
\expandafter\def\csname PY@tok@cm\endcsname{\let\PY@it=\textit\def\PY@tc##1{\textcolor[rgb]{0.25,0.50,0.50}{##1}}}
\expandafter\def\csname PY@tok@cp\endcsname{\def\PY@tc##1{\textcolor[rgb]{0.74,0.48,0.00}{##1}}}
\expandafter\def\csname PY@tok@vg\endcsname{\def\PY@tc##1{\textcolor[rgb]{0.10,0.09,0.49}{##1}}}

\def\PYZbs{\char`\\}
\def\PYZus{\char`\_}
\def\PYZob{\char`\{}
\def\PYZcb{\char`\}}
\def\PYZca{\char`\^}
\def\PYZam{\char`\&}
\def\PYZlt{\char`\<}
\def\PYZgt{\char`\>}
\def\PYZsh{\char`\#}
\def\PYZpc{\char`\%}
\def\PYZdl{\char`\$}
\def\PYZhy{\char`\-}
\def\PYZsq{\char`\'}
\def\PYZdq{\char`\"}
\def\PYZti{\char`\~}
% for compatibility with earlier versions
\def\PYZat{@}
\def\PYZlb{[}
\def\PYZrb{]}
\makeatother


    % Exact colors from NB
    \definecolor{incolor}{rgb}{0.0, 0.0, 0.5}
    \definecolor{outcolor}{rgb}{0.545, 0.0, 0.0}



    
    % Prevent overflowing lines due to hard-to-break entities
    \sloppy 
    % Setup hyperref package
    \hypersetup{
      breaklinks=true,  % so long urls are correctly broken across lines
      colorlinks=true,
      urlcolor=blue,
      linkcolor=darkorange,
      citecolor=darkgreen,
      }
    % Slightly bigger margins than the latex defaults
    
    \geometry{verbose,tmargin=1in,bmargin=1in,lmargin=1in,rmargin=1in}
    
    

    \begin{document}
    
    
    \maketitle
    
    

    
    \section{Exploring the Lorenz System of Differential
Equations}\label{exploring-the-lorenz-system-of-differential-equations}

    In this Notebook we explore the Lorenz system of differential equations:

\[
\begin{aligned}
\dot{x} & = \sigma(y-x) \\
\dot{y} & = \rho x - y - xz \\
\dot{z} & = -\beta z + xy
\end{aligned}
\]

This is one of the classic systems in non-linear differential equations.
It exhibits a range of different behaviors as the parameters ($\sigma$,
$\beta$, $\rho$) are varied.

    \subsection{Imports}\label{imports}

    First, we import the needed things from IPython, NumPy, Matplotlib and
SciPy.

    \begin{Verbatim}[commandchars=\\\{\}]
{\color{incolor}In [{\color{incolor}1}]:} \PY{o}{\PYZpc{}}\PY{k}{matplotlib} inline
\end{Verbatim}

    \begin{Verbatim}[commandchars=\\\{\}]
{\color{incolor}In [{\color{incolor}2}]:} \PY{k+kn}{from} \PY{n+nn}{IPython}\PY{n+nn}{.}\PY{n+nn}{html}\PY{n+nn}{.}\PY{n+nn}{widgets} \PY{k}{import} \PY{n}{interact}\PY{p}{,} \PY{n}{interactive}
        \PY{k+kn}{from} \PY{n+nn}{IPython}\PY{n+nn}{.}\PY{n+nn}{display} \PY{k}{import} \PY{n}{clear\PYZus{}output}\PY{p}{,} \PY{n}{display}\PY{p}{,} \PY{n}{HTML}
\end{Verbatim}

    \begin{Verbatim}[commandchars=\\\{\}]
:0: FutureWarning: IPython widgets are experimental and may change in the future.
    \end{Verbatim}

    \begin{Verbatim}[commandchars=\\\{\}]
{\color{incolor}In [{\color{incolor}3}]:} \PY{k+kn}{import} \PY{n+nn}{numpy} \PY{k}{as} \PY{n+nn}{np}
        \PY{k+kn}{from} \PY{n+nn}{scipy} \PY{k}{import} \PY{n}{integrate}
        
        \PY{k+kn}{from} \PY{n+nn}{matplotlib} \PY{k}{import} \PY{n}{pyplot} \PY{k}{as} \PY{n}{plt}
        \PY{k+kn}{from} \PY{n+nn}{mpl\PYZus{}toolkits}\PY{n+nn}{.}\PY{n+nn}{mplot3d} \PY{k}{import} \PY{n}{Axes3D}
        \PY{k+kn}{from} \PY{n+nn}{matplotlib}\PY{n+nn}{.}\PY{n+nn}{colors} \PY{k}{import} \PY{n}{cnames}
        \PY{k+kn}{from} \PY{n+nn}{matplotlib} \PY{k}{import} \PY{n}{animation}
\end{Verbatim}

    \subsection{Computing the trajectories and plotting the
result}\label{computing-the-trajectories-and-plotting-the-result}

    We define a function that can integrate the differential equations
numerically and then plot the solutions. This function has arguments
that control the parameters of the differential equation ($\sigma$,
$\beta$, $\rho$), the numerical integration (\texttt{N},
\texttt{max\_time}) and the visualization (\texttt{angle}).

    \begin{Verbatim}[commandchars=\\\{\}]
{\color{incolor}In [{\color{incolor}4}]:} \PY{k}{def} \PY{n+nf}{solve\PYZus{}lorenz}\PY{p}{(}\PY{n}{N}\PY{o}{=}\PY{l+m+mi}{10}\PY{p}{,} \PY{n}{angle}\PY{o}{=}\PY{l+m+mf}{0.0}\PY{p}{,} \PY{n}{max\PYZus{}time}\PY{o}{=}\PY{l+m+mf}{4.0}\PY{p}{,} \PY{n}{sigma}\PY{o}{=}\PY{l+m+mf}{10.0}\PY{p}{,} \PY{n}{beta}\PY{o}{=}\PY{l+m+mf}{8.}\PY{o}{/}\PY{l+m+mi}{3}\PY{p}{,} \PY{n}{rho}\PY{o}{=}\PY{l+m+mf}{28.0}\PY{p}{)}\PY{p}{:}
        
            \PY{n}{fig} \PY{o}{=} \PY{n}{plt}\PY{o}{.}\PY{n}{figure}\PY{p}{(}\PY{p}{)}
            \PY{n}{ax} \PY{o}{=} \PY{n}{fig}\PY{o}{.}\PY{n}{add\PYZus{}axes}\PY{p}{(}\PY{p}{[}\PY{l+m+mi}{0}\PY{p}{,} \PY{l+m+mi}{0}\PY{p}{,} \PY{l+m+mi}{1}\PY{p}{,} \PY{l+m+mi}{1}\PY{p}{]}\PY{p}{,} \PY{n}{projection}\PY{o}{=}\PY{l+s}{\PYZsq{}}\PY{l+s}{3d}\PY{l+s}{\PYZsq{}}\PY{p}{)}
            \PY{n}{ax}\PY{o}{.}\PY{n}{axis}\PY{p}{(}\PY{l+s}{\PYZsq{}}\PY{l+s}{off}\PY{l+s}{\PYZsq{}}\PY{p}{)}
        
            \PY{c}{\PYZsh{} prepare the axes limits}
            \PY{n}{ax}\PY{o}{.}\PY{n}{set\PYZus{}xlim}\PY{p}{(}\PY{p}{(}\PY{o}{\PYZhy{}}\PY{l+m+mi}{25}\PY{p}{,} \PY{l+m+mi}{25}\PY{p}{)}\PY{p}{)}
            \PY{n}{ax}\PY{o}{.}\PY{n}{set\PYZus{}ylim}\PY{p}{(}\PY{p}{(}\PY{o}{\PYZhy{}}\PY{l+m+mi}{35}\PY{p}{,} \PY{l+m+mi}{35}\PY{p}{)}\PY{p}{)}
            \PY{n}{ax}\PY{o}{.}\PY{n}{set\PYZus{}zlim}\PY{p}{(}\PY{p}{(}\PY{l+m+mi}{5}\PY{p}{,} \PY{l+m+mi}{55}\PY{p}{)}\PY{p}{)}
            
            \PY{k}{def} \PY{n+nf}{lorenz\PYZus{}deriv}\PY{p}{(}\PY{n}{x\PYZus{}y\PYZus{}z}\PY{p}{,} \PY{n}{t0}\PY{p}{,} \PY{n}{sigma}\PY{o}{=}\PY{n}{sigma}\PY{p}{,} \PY{n}{beta}\PY{o}{=}\PY{n}{beta}\PY{p}{,} \PY{n}{rho}\PY{o}{=}\PY{n}{rho}\PY{p}{)}\PY{p}{:}
                \PY{l+s+sd}{\PYZdq{}\PYZdq{}\PYZdq{}Compute the time\PYZhy{}derivative of a Lorenz system.\PYZdq{}\PYZdq{}\PYZdq{}}
                \PY{n}{x}\PY{p}{,} \PY{n}{y}\PY{p}{,} \PY{n}{z} \PY{o}{=} \PY{n}{x\PYZus{}y\PYZus{}z}
                \PY{k}{return} \PY{p}{[}\PY{n}{sigma} \PY{o}{*} \PY{p}{(}\PY{n}{y} \PY{o}{\PYZhy{}} \PY{n}{x}\PY{p}{)}\PY{p}{,} \PY{n}{x} \PY{o}{*} \PY{p}{(}\PY{n}{rho} \PY{o}{\PYZhy{}} \PY{n}{z}\PY{p}{)} \PY{o}{\PYZhy{}} \PY{n}{y}\PY{p}{,} \PY{n}{x} \PY{o}{*} \PY{n}{y} \PY{o}{\PYZhy{}} \PY{n}{beta} \PY{o}{*} \PY{n}{z}\PY{p}{]}
        
            \PY{c}{\PYZsh{} Choose random starting points, uniformly distributed from \PYZhy{}15 to 15}
            \PY{n}{np}\PY{o}{.}\PY{n}{random}\PY{o}{.}\PY{n}{seed}\PY{p}{(}\PY{l+m+mi}{1}\PY{p}{)}
            \PY{n}{x0} \PY{o}{=} \PY{o}{\PYZhy{}}\PY{l+m+mi}{15} \PY{o}{+} \PY{l+m+mi}{30} \PY{o}{*} \PY{n}{np}\PY{o}{.}\PY{n}{random}\PY{o}{.}\PY{n}{random}\PY{p}{(}\PY{p}{(}\PY{n}{N}\PY{p}{,} \PY{l+m+mi}{3}\PY{p}{)}\PY{p}{)}
        
            \PY{c}{\PYZsh{} Solve for the trajectories}
            \PY{n}{t} \PY{o}{=} \PY{n}{np}\PY{o}{.}\PY{n}{linspace}\PY{p}{(}\PY{l+m+mi}{0}\PY{p}{,} \PY{n}{max\PYZus{}time}\PY{p}{,} \PY{n+nb}{int}\PY{p}{(}\PY{l+m+mi}{250}\PY{o}{*}\PY{n}{max\PYZus{}time}\PY{p}{)}\PY{p}{)}
            \PY{n}{x\PYZus{}t} \PY{o}{=} \PY{n}{np}\PY{o}{.}\PY{n}{asarray}\PY{p}{(}\PY{p}{[}\PY{n}{integrate}\PY{o}{.}\PY{n}{odeint}\PY{p}{(}\PY{n}{lorenz\PYZus{}deriv}\PY{p}{,} \PY{n}{x0i}\PY{p}{,} \PY{n}{t}\PY{p}{)}
                              \PY{k}{for} \PY{n}{x0i} \PY{o+ow}{in} \PY{n}{x0}\PY{p}{]}\PY{p}{)}
            
            \PY{c}{\PYZsh{} choose a different color for each trajectory}
            \PY{n}{colors} \PY{o}{=} \PY{n}{plt}\PY{o}{.}\PY{n}{cm}\PY{o}{.}\PY{n}{jet}\PY{p}{(}\PY{n}{np}\PY{o}{.}\PY{n}{linspace}\PY{p}{(}\PY{l+m+mi}{0}\PY{p}{,} \PY{l+m+mi}{1}\PY{p}{,} \PY{n}{N}\PY{p}{)}\PY{p}{)}
        
            \PY{k}{for} \PY{n}{i} \PY{o+ow}{in} \PY{n+nb}{range}\PY{p}{(}\PY{n}{N}\PY{p}{)}\PY{p}{:}
                \PY{n}{x}\PY{p}{,} \PY{n}{y}\PY{p}{,} \PY{n}{z} \PY{o}{=} \PY{n}{x\PYZus{}t}\PY{p}{[}\PY{n}{i}\PY{p}{,}\PY{p}{:}\PY{p}{,}\PY{p}{:}\PY{p}{]}\PY{o}{.}\PY{n}{T}
                \PY{n}{lines} \PY{o}{=} \PY{n}{ax}\PY{o}{.}\PY{n}{plot}\PY{p}{(}\PY{n}{x}\PY{p}{,} \PY{n}{y}\PY{p}{,} \PY{n}{z}\PY{p}{,} \PY{l+s}{\PYZsq{}}\PY{l+s}{\PYZhy{}}\PY{l+s}{\PYZsq{}}\PY{p}{,} \PY{n}{c}\PY{o}{=}\PY{n}{colors}\PY{p}{[}\PY{n}{i}\PY{p}{]}\PY{p}{)}
                \PY{n}{plt}\PY{o}{.}\PY{n}{setp}\PY{p}{(}\PY{n}{lines}\PY{p}{,} \PY{n}{linewidth}\PY{o}{=}\PY{l+m+mi}{2}\PY{p}{)}
        
            \PY{n}{ax}\PY{o}{.}\PY{n}{view\PYZus{}init}\PY{p}{(}\PY{l+m+mi}{30}\PY{p}{,} \PY{n}{angle}\PY{p}{)}
            \PY{n}{plt}\PY{o}{.}\PY{n}{show}\PY{p}{(}\PY{p}{)}
        
            \PY{k}{return} \PY{n}{t}\PY{p}{,} \PY{n}{x\PYZus{}t}
\end{Verbatim}

    Let's call the function once to view the solutions. For this set of
parameters, we see the trajectories swirling around two points, called
attractors.

    \begin{Verbatim}[commandchars=\\\{\}]
{\color{incolor}In [{\color{incolor}5}]:} \PY{n}{t}\PY{p}{,} \PY{n}{x\PYZus{}t} \PY{o}{=} \PY{n}{solve\PYZus{}lorenz}\PY{p}{(}\PY{n}{angle}\PY{o}{=}\PY{l+m+mi}{0}\PY{p}{,} \PY{n}{N}\PY{o}{=}\PY{l+m+mi}{10}\PY{p}{)}
\end{Verbatim}

    \begin{center}
    \adjustimage{max size={0.9\linewidth}{0.9\paperheight}}{Lorenz%20Differential%20Equations_files/Lorenz%20Differential%20Equations_11_0.png}
    \end{center}
    { \hspace*{\fill} \\}
    
    Using IPython's \texttt{interactive} function, we can explore how the
trajectories behave as we change the various parameters.

    \begin{Verbatim}[commandchars=\\\{\}]
{\color{incolor}In [{\color{incolor}6}]:} \PY{n}{w} \PY{o}{=} \PY{n}{interactive}\PY{p}{(}\PY{n}{solve\PYZus{}lorenz}\PY{p}{,} \PY{n}{max\PYZus{}time}\PY{o}{=}\PY{p}{(}\PY{o}{\PYZhy{}}\PY{l+m+mf}{4.0}\PY{p}{,}\PY{l+m+mf}{20.0}\PY{p}{)}\PY{p}{,} \PY{n}{angle}\PY{o}{=}\PY{p}{(}\PY{l+m+mf}{0.}\PY{p}{,}\PY{l+m+mf}{360.}\PY{p}{)}\PY{p}{,} \PY{n}{N}\PY{o}{=}\PY{p}{(}\PY{l+m+mi}{10}\PY{p}{,}\PY{l+m+mi}{40}\PY{p}{)}\PY{p}{,} \PY{n}{sigma}\PY{o}{=}\PY{p}{(}\PY{l+m+mf}{0.0}\PY{p}{,}\PY{l+m+mf}{50.0}\PY{p}{)}\PY{p}{,} \PY{n}{rho}\PY{o}{=}\PY{p}{(}\PY{l+m+mf}{0.0}\PY{p}{,}\PY{l+m+mf}{50.0}\PY{p}{)}\PY{p}{)}
        \PY{n}{display}\PY{p}{(}\PY{n}{w}\PY{p}{)}
\end{Verbatim}

    \begin{center}
    \adjustimage{max size={0.9\linewidth}{0.9\paperheight}}{Lorenz%20Differential%20Equations_files/Lorenz%20Differential%20Equations_13_0.png}
    \end{center}
    { \hspace*{\fill} \\}
    
    The object returned by \texttt{interactive} is a \texttt{Widget} object
and it has attributes that contain the current result and arguments:

    \begin{Verbatim}[commandchars=\\\{\}]
{\color{incolor}In [{\color{incolor}7}]:} \PY{n}{t}\PY{p}{,} \PY{n}{x\PYZus{}t} \PY{o}{=} \PY{n}{w}\PY{o}{.}\PY{n}{result}
\end{Verbatim}

    \begin{Verbatim}[commandchars=\\\{\}]
{\color{incolor}In [{\color{incolor}8}]:} \PY{n}{w}\PY{o}{.}\PY{n}{kwargs}
\end{Verbatim}

            \begin{Verbatim}[commandchars=\\\{\}]
{\color{outcolor}Out[{\color{outcolor}8}]:} \{'N': 10,
         'angle': 0.0,
         'beta': 2.6666666666666665,
         'max\_time': 4.0,
         'rho': 28.0,
         'sigma': 10.0\}
\end{Verbatim}
        
    After interacting with the system, we can take the result and perform
further computations. In this case, we compute the average positions in
$x$, $y$ and $z$.

    \begin{Verbatim}[commandchars=\\\{\}]
{\color{incolor}In [{\color{incolor}9}]:} \PY{n}{xyz\PYZus{}avg} \PY{o}{=} \PY{n}{x\PYZus{}t}\PY{o}{.}\PY{n}{mean}\PY{p}{(}\PY{n}{axis}\PY{o}{=}\PY{l+m+mi}{1}\PY{p}{)}
\end{Verbatim}

    \begin{Verbatim}[commandchars=\\\{\}]
{\color{incolor}In [{\color{incolor}10}]:} \PY{n}{xyz\PYZus{}avg}\PY{o}{.}\PY{n}{shape}
\end{Verbatim}

            \begin{Verbatim}[commandchars=\\\{\}]
{\color{outcolor}Out[{\color{outcolor}10}]:} (10, 3)
\end{Verbatim}
        
    Creating histograms of the average positions (across different
trajectories) show that on average the trajectories swirl about the
attractors.

    \begin{Verbatim}[commandchars=\\\{\}]
{\color{incolor}In [{\color{incolor}11}]:} \PY{n}{plt}\PY{o}{.}\PY{n}{hist}\PY{p}{(}\PY{n}{xyz\PYZus{}avg}\PY{p}{[}\PY{p}{:}\PY{p}{,}\PY{l+m+mi}{0}\PY{p}{]}\PY{p}{)}
         \PY{n}{plt}\PY{o}{.}\PY{n}{title}\PY{p}{(}\PY{l+s}{\PYZsq{}}\PY{l+s}{Average \PYZdl{}x(t)\PYZdl{}}\PY{l+s}{\PYZsq{}}\PY{p}{)}
\end{Verbatim}

            \begin{Verbatim}[commandchars=\\\{\}]
{\color{outcolor}Out[{\color{outcolor}11}]:} <matplotlib.text.Text at 0x111b7af98>
\end{Verbatim}
        
    \begin{center}
    \adjustimage{max size={0.9\linewidth}{0.9\paperheight}}{Lorenz%20Differential%20Equations_files/Lorenz%20Differential%20Equations_21_1.png}
    \end{center}
    { \hspace*{\fill} \\}
    
    \begin{Verbatim}[commandchars=\\\{\}]
{\color{incolor}In [{\color{incolor}12}]:} \PY{n}{plt}\PY{o}{.}\PY{n}{hist}\PY{p}{(}\PY{n}{xyz\PYZus{}avg}\PY{p}{[}\PY{p}{:}\PY{p}{,}\PY{l+m+mi}{1}\PY{p}{]}\PY{p}{)}
         \PY{n}{plt}\PY{o}{.}\PY{n}{title}\PY{p}{(}\PY{l+s}{\PYZsq{}}\PY{l+s}{Average \PYZdl{}y(t)\PYZdl{}}\PY{l+s}{\PYZsq{}}\PY{p}{)}
\end{Verbatim}

            \begin{Verbatim}[commandchars=\\\{\}]
{\color{outcolor}Out[{\color{outcolor}12}]:} <matplotlib.text.Text at 0x111d80710>
\end{Verbatim}
        
    \begin{center}
    \adjustimage{max size={0.9\linewidth}{0.9\paperheight}}{Lorenz%20Differential%20Equations_files/Lorenz%20Differential%20Equations_22_1.png}
    \end{center}
    { \hspace*{\fill} \\}
    
    \begin{Verbatim}[commandchars=\\\{\}]
{\color{incolor}In [{\color{incolor} }]:} 
\end{Verbatim}

    \begin{Verbatim}[commandchars=\\\{\}]
{\color{incolor}In [{\color{incolor} }]:} 
\end{Verbatim}


    % Add a bibliography block to the postdoc
    
    
    
    \end{document}
